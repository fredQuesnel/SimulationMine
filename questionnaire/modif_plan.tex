\section{Modification du plan d'opération}

Le plan d'opérations peut être modifié dans le but d'atteindre divers objectifs de production. Pour modifier la cible d'une pelle, effectuez un clic droit sur la pelle, cliquez sur "Modifier le plan", et entrez la valeur désirée. Pour remettre les valeurs par défaut, chargez la mine de nouveau.

Protocole expérimental 

•Mine : 10 pelles

•Nombre de camions de 60 tonnes: 20

•Nombre de camions de 100 tonnes: 20

•Temps de simulation : 24h

•Fonction de score : optimise ou optimalassign.

\subsection{Estimation de production}

Supposez que le plan par défaut est suivi à la lettre. Estimez quelle devrait être, sur une période de 24h : 

•La quantité de minerai livrée au concentrateur.

•La quantité de stérile livrée à la pile de stérile.

•Le pourcentage de fer et de soufre au concentrateur.

Comparez ces chiffres avec les résultats d'une simulation. Comment expliquez-vous la différence observée?

\subsection{Modification manuelle du plan}

Un nouveau client souhaite acheter du minerai de très bonne qualité (et à payer cher pour!). Il lui faut au moins X tonnes de minerais d'ici 24 heures! Les nouveaux objectifs et contraintes de production sont : 

•Objectif : Maximiser (on priorise davantage la production de minerai).

•Quantité de minerai : 

•Taux de fer : 

•Taux de soufre : 

Proposez un plan d'opération qui permet de respecter toutes les contraintes (dans une simulation). Simulez ce plan et donnez-en les caractéristiques de production.

\subsection{Optimisation du plan}

En supposant que n'importe quel plan peut être suivi à la lettre, écrivez un modèle d'optimisation linéaire qui détermine le meilleur plan d'opérations en régime continu. Utilisez les mêmes objectifs et contraintes que pour la section 

À l'aide d'un logiciel d'optimisation (le solveur d'Excel, par exemple), résolvez ce problème et donnez sa solution. Selon votre plan, estimez :

•La quantité de minerai livrée au concentrateur.

•La quantité de stérile livrée à la pile de stérile.

•Le pourcentage de fer et de soufre au concentrateur.

Plan "optimal" en pratique

Simulez le plan trouvé en 4.2. 

Dans votre réponse, 

•Comparez les résultats de la simulation avec vos prédictions de la question . Expliquez les différences observées ( autrement dit, quels sont les éléments que le modèle d'optimisation linéaire ne considère pas?).

•Les modèles linéaires ont tendance à produire des solutions extrêmes. Expliquez en quoi cela pourrait être un problème avec le modèle développé en .

\subsection{Ajustement dynamique du plan}

Il arrive qu'une pelle subisse un bris mécanique et tombe en panne pour une durée indéterminée. Il faut alors ajuster le plan afin de maintenir le respect des objectifs de production. Lorsqu'une pelle redevient fonctionnelle, il faut à nouveau réajuster le plan. Il se peut également que plusieurs pelles soient en panne simultanément.

Vous devez développer une stratégie de mise à jour du plan. Pour ce faire, modifiez les fonctions et de la classe .

Testez votre stratégie en simulant 10 jours sur la mine 10 pelles. Utilisez le problème d'affectation pour assigner les camions (option optimiseplan).

Il faut simuler 10 jours pour qu'il y ait assez de pannes pour que les résultats soient statistiquement significatifs.

Dans votre réponse, 

•Décrivez brièvement votre stratégie de mise à jour du plan.

•Donnez les résultats pertinents de la simulation.

•Comparez vos résultats avec une simulation de 10 jours avec un plan fixe. 

Bonus (?) : Si la stratégie utilise un modèle de programmation linéaire.
